
\section{Dictionaries}

Data storage is another topic that we should touch upon. We may 
need a global ``configuration'' data space, that controls options
for the entire program. Also, individual plugins may have configuration
data that changes their behavior. Plugins may also need to record 
some data for use by other plugins.

Included is a dictionary class to satisfy these storage needs. Each
entry has a name string and a data object. Dictionaries end up 
being used in a variety of places. First, {\tt PluginRegistry} 
holds a global dictionary that each plugin contained within can
access. This is useful for global options, e.g. ``debug=true''. Further,
each {\tt PluginChain} holds a dictionary for 
per-chain configuration data (See Section \ref{sec:Plugin Chains}).

Perhaps it is best to look at an example of the dictionary usage at this point:

\begin{lstlisting}[frame=single]
 Dict          d;
 BoolDictItem *boolItem = new BoolDictItem;

 boolItem->set(true);
 d.set("debug", item);
\end{lstlisting}

Here, we see the separation between dictionary and data object.
Data objects ({\tt DictItem}) are a bit of a tricky case. We could
try and make them a variant class or a union, but that makes it
difficult to extend. Further, there's an issue of ownership in
the above example - who is supposed to free the DictItems?

First, we'll deal with the ownership issue. Currently, allocation
and deallocation of {\tt DictItem} classes should be done through
the {\tt PluginRegistry}. Why there - because plugins may extend
the {\tt DictItem} class. This requires the allocation to be done
in the plugins, and since the {\\ PluginRegistry} is the only one 
to know about plugin entry points, that's where the item allocation
ends up.

To be somewhat more correct, the previous example should look
something more like:

\begin{lstlisting}[frame=single]
 void foo(PluginRegistry *registry) 
 {
   Dict          d(registry);
   BoolDictItem *boolItem;

   if (registry == NULL) return;

   boolItem = registry->createDictItem("bool");

   boolItem->set(true);
   d.set("debug", item);
 }
\end{lstlisting}

Here, we're allocating things appropriately. Also in this example, we
need to inform a {\tt Dict} about the registry, so it can deallocate items
that it's holding when the {\tt Dict} is destroyed.

Currently, there is a small list of built-in data types:

\begin{center}
\begin{tabular}{l|l}
Argument            & Data type \\ \hline
{\tt "bool"}        & {\tt BoolDictItem}   \\ \hline
{\tt "int"}         &  {\tt IntDictItem}     \\ \hline
{\tt "double"}      &  {\tt DoubleDictItem}     \\ \hline
{\tt "string"}      &  {\tt StringDictItem}     \\ \hline
{\tt "blob"}        &  {\tt BlobDictItem}    \\ \hline
{\tt "intArray"}    &  {\tt IntArrayDictItem}     \\ \hline
{\tt "doubleArray"} &  {\tt DoubleArrayDictItem}     \\ \hline
{\tt "stringArray"} &  {\tt StringArrayDictItem}     \\ \hline
\end{tabular}
\end{center}


\subsection{Extending {\tt DictItem}}

Plugins can create their own classes derived from {\tt DictItem} and expose
them for others to use. This can be done in a similar manner to plugin 
registration:

\begin{lstlisting}[frame=single]
 class MyDictItem: public DictItem
 {
    ...
 };

 class foo: public Plugin
 {
   foo();
   virtual ~foo();

   // NOTE: NO USE OF PLUGIN_ALLOC_FUNCS!

   BEGIN_DICTITEM_ALLOC_FUNCS(foo)
     DICTITEM_ALLOC("cool_item", MyDictItem)
   END_DICTITEM_ALLOC_FUNCS
   BEGIN_DICTITEM_DELETE_FUNCS
     DICTITEM_DELETE("cool_item")
   END_DICTITEM_DELETE_FUNCS
 }

\end{lstlisting}

Here, "cool\_item" is the string to passed into {\tt PluginRegistry::createDictItem} and
is also the string returned by {\tt MyDictItem::itemType()}.




\subsection{Saving and Loading}

Both the {\tt Dict} and {\tt DictItem} classes have serialization and unserialization
methods that can be used to load and save data. In both cases, they make use of 
the {\tt libxml2}. Data will be stored in a form:

\lstset{language=XML,
		columns=fixed,
        showspaces=false,
        showstringspaces=false,
        basicstyle=\ttfamily,
		backgroundcolor=\color{backGray}}
\begin{lstlisting}[frame=single]
 <dict name="Test Dict 1">
     <dictitem type="bool" name="Debug">true</dictitem>

     <dictitem type="double vector" name="measured white Yxy">
         <value>80.012</value>
         <value>0.3333</value>
         <value>0.3333</value>
     </dictitem>
 </dict>
\end{lstlisting}

Saving a {\tt Dict} is fairly straightfoward, we just need the root node where
we would like to insert the {\tt Dict}. For example:


\subsection{Managing a {\tt Dict}}

It would be nice if we were able to share {\tt Dict} data with everyone. 
This way, plugins could communicate with one another, or with the
outside world, by exchanging {\tt Dict}s. This is the motivation of the
{\tt DictHash} plugin. It is a built-in plugin which should hold all 
the {\tt Dict} objects that are created. It has a fairly simple interface:

\lstset{language=c++,
		columns=fixed,
        basicstyle=\ttfamily,
		backgroundcolor=\color{backGray}}


\begin{description}
\item[] \
\begin{lstlisting}[frame=single]
Dict * DictHash::newDict(std::string name);
\end{lstlisting}
This will create a new {\tt Dict} with the given name. If something already
exists with this name, don't remove it, but rather just return a pointer.
If we forget to name multiple {\tt Dict}s, we don't want to loose data
by over-writing.

\item[] \
\begin{lstlisting}[frame=single]
bool DictHash::clearDict(std::string name);
\end{lstlisting}
Remove a {\tt Dict} from our storage.

\item[] \
\begin{lstlisting}[frame=single]
Dict * DictHash::getDict(std::string name);
\end{lstlisting}
Lookup a stored {\tt Dict}. Returns NULL if we don't have
anything stored by that name. You shouldn't hold the pointer 
that is returned, but instead re-query every time you 
need the reference within reason.

\item[] \
\begin{lstlisting}[frame=single]
std::vector<std::string> DictHash::getDictNames();
\end{lstlisting}
Return a vector of all the names of {\tt Dict}s that we hold.
\end{description}


