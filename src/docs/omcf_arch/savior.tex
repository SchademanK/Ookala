
\section{Saving Data}

Thus far, we have touched on configuration files for plugin chains.
These files are read-only, as far as the application is concerned. 
This then begs the question of how to persist data to disk. There
are a variety of potentially interesting pieces to persist, such
as the time of last calibration, display readings used to compute
lookup tables, and so forth.

Enter the {\tt DataSavior} plugin. This is just a dictionary with
some associated file on disk that can be loaded and saved.  The
dictionary can be accessed through {\tt DataSavior::mDict}.

The on-disk files to use are configured through a vector of strings
in the global dictionary named ``DataSavior Files''. File names
can contain a {\bf \%s} which is substituted for the current 
host name. Multiple files can be specified.

On {\tt DataSavior::load()} and in {\tt DataSavior::\_postLoad()}, 
the contents of the {\em newest} data file are read into the dictionary. 

On {\tt DataSavior::save()}, the contents of the dictionary are 
saved to {\em all} data files.

This strategy was chosen to allow for storing data in multiple places
for redundancy, should one of the locations not be available. For 
example, one file could live on a networked file system while another
lives on the local file system. Should the network be unavailable,
we should still be able to save locally. Then, when loading, if
the network was available, the local file would be chosen as it 
has a later modification time.

