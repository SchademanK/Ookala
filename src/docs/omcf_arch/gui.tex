\section{Plugins and Guis}

A rather important component is a gui. For monitor calibration,
this is the place for displaying windows with big color patches
for measurement. They also give a chance to provide the user 
with feedback about what is going on in the calibration process.

\subsection{\tt GuiBase}

In order to make gui elements more independent from their users, 
we have attempted to put virtual methods into a base class for 
gui plugins, aptly named {\tt GuiBase}. 

There are two pure-virtual methods, {\tt GuiBase::showUi()} and
{\tt GuiBase::setMeasureColor()}. The former controls hiding and
showing the main gui window, while the latter controls the 
colored measurement patch.

Assembling a gui base class allows plugins to query for a gui and
not need to know its class. It doesn't discount the possibility, 
however, of locking in a particular plugin to only work with 
a particular gui (by querying by the gui plugin name). The typical
way for a plugin to access the gui would be something like the
following:

\begin{lstlisting}[frame=single]
 bool
 MyPlugin::_run(PluginChain *chain) 
 {
     Plugin  *pi;
     GuiBase *gui; 

     // This should really be done earlier, in _preRun()
     pi = chain->queryByAttribute("gui");
     if (pi == NULL) {
         return false;
     }

     // Make sure we really have a gui plugin
     gui = dynamic_cast<GuiBase *>(pi);
     if (gui == NULL) {
         return false;
     }

     gui->setMeasureColor(255, 255, 255);
 }
\end{lstlisting}


\subsection{Plugin Gui Methods}

\begin{description}
\item[] \
\begin{lstlisting}[frame=single]
 virtual bool Plugin::guiStart()
 virtual bool Plugin::_guiStart()
\end{lstlisting}

This will be called when the App starts up. It should be the
first place where out app pointer is value. Do any sort of
GUI setup here (actually, do it in {\tt \_guiStart()})

For example, our Gui-providing plugins would use this to
setup their main windows, but not show them yet [that would
be done through {\tt GuiBase::show()}].


\item[] \
\begin{lstlisting}[frame=single]
 virtual bool Plugin::guiStop()
 virtual bool Plugin::_guiStop()
\end{lstlisting}

This will be called when the app shuts down. Destroy any
GUI elements that you created in {\tt guiStart()} (but put the
code you write in {\tt \_guiStop()}).


\item[] \
\begin{lstlisting}[frame=single]
 virtual bool Plugin::buildParamUi(wxSizer *sizer, PluginChain *chain)
 virtual bool Plugin::_buildParamUi(wxSizer *sizer, PluginChain *chain)
\end{lstlisting}

This will be used to build a gui for manual parameter control.
When this happens, it will loop over all the plugins in a chain,
allowing them to add whatever widetry into a sizer (which is probably
a wxWidgets vertical box).

If you don't add any Gui components, return {\tt false}. Then we know
to not display the window that is being build if it dones't have
anything interesting in it.
But, if you add something, return {\tt true}.
 
{\tt buildParamUi} is called as part of a {\tt PluginChain} execution. 

\item[] \
\begin{lstlisting}[frame=single]
 virtual bool Plugin::processEvents()
\end{lstlisting}

Since we are not spawning a separate thread for the gui main loop, we need
to take care to handle events periodically to keep the gui from becoming
unresponsive.  This method will enter into the wxWidgets main loop and 
process whatever events are pending. Plugins that are going to run for 
a while should periodically call this in order to handle events. Note that
this may cause execution flow to travel elsewhere for a period.

You probably do not want to override this method.


\end{description}




