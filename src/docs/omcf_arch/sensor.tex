\section{Color Sensor Plugins}

Simliar to the motivation for a base class for gui's, we also 
have a base class for color probes. This way, a plugin that 
needs color measurement capabilities can do so independently 
of which probes are attached (and have appropriate plugins loaded).

\subsection{\tt Sensor}

Sensors may be spectrometers or colorimeters, or things of that nature.
We've still named the 
base class {\tt Sensor}, for ease of typing (But if you have 
a better name - speak now!).

{\tt Sensor} has three pure-virtual methods: 
{\tt \_init()}, {\tt \_shutdown()}, and {\tt \_measureYxy()}

\begin{description}

\item[] \
\begin{lstlisting}[frame=single]
 virtual bool Sensor::init()
 virtual bool Sensor::_init() = 0;
\end{lstlisting}

Most devices require some sort of one-time setup. We'll
setup things such that if this hasn't happened by the 
time we make our first measurement, we'll run {\tt init()}.

{\tt init()} is different from {\tt postLoad()}. {\tt postLoad()}
should
just test for the existence of a device, and fail if one
is not found. We need sometime between {\tt postLoad()} and
the first time we measure where we can config the device.
This could be done in {\tt postCheckDeps()}, but we still 
need to allow for some configuration before starting up the 
probe.


\item[] \
\begin{lstlisting}[frame=single]
 virtual bool Sensor::shutdown()
 virtual bool Sensor::_shutdown() = 0;
\end{lstlisting}

Does any one-time shutdown procedures needed to cleanup
after the device.


\item[] \
\begin{lstlisting}[frame=single]
 virtual bool Sensor::measureYxy(double *valueYxy)
 virtual bool Sensor::_measureYxy(double *valueYxy) = 0;
\end{lstlisting}

Sample from a device to read a value. Results should
come back as Yxy in the desired units. 

Its a good practice to call {\tt init()} before {\tt measureYxy()},     
as {\tt init()} may need to do its own measuring or manipulating
of the display output. Failing to do so may be an error.

The version you want to implement is {\tt \_measureYxy()}.

{\tt measureYxy()} assumes that your {\tt \_measureYxy()} will block while
it is reading. So, we'll start up a thread in {\tt measureYxy()} to call
{\tt \_measureYxy()} and handle main loop events while we wait.


\item[] \
\begin{lstlisting}[frame=single]
 virtual bool Sensor::setUnits(const std::string units)
 const std::string Sensor::getUnits()
\end{lstlisting}

Set the desired measurement units. This should be done prior
to {\tt init()}. Common options will include:
\begin{itemize}
     \item ``fL''              
     \item ``cd/m\^\ 2'' or ``nits''
\end{itemize}

The parameters are strings to 
avoid enum fighting from people wanting to use non-standard values.
This does mean that you should be somewhat careful to check 
return values.

\item[] \
\begin{lstlisting}[frame=single]
 virtual bool Sensor::setDisplayTech(const std::string display)
 const std::string Sensor::getDisplayTech()
\end{lstlisting}

Set the display technology. This should be done prior to
{\tt init()}. Common options will include things like:
\begin{itemize}
    \item ``crt''
    \item ``lcd''
    \item ``led crt''
    \item ``dcinema dlp''
\end{itemize}

The parameters are strings to 
avoid enum fighting from people wanting to use non-standard values.
This does mean that you should be somewhat careful to check 
return values.

\item[] \
\begin{lstlisting}[frame=single]
 bool Sensor::setIntegrationTime(const std::string length);
 const std::string Sensor::getIntegrationTime();
\end{lstlisting}

Set the integration time, based on a category. If you have
setup to measure a CRT and choose AUTO mode here, you should
be prepared for the colorimeter to read the screen for
refersh rates during {\tt init()}, which may require cranking 
the luminance output of the display (which you should do
prior to calling {\tt init()} since the colorimeter doesn't really
know which display you are using).
Common options will include:
\begin{itemize}
    \item ``auto''
    \item ``fast''
    \item ``slow''
\end{itemize}

The parameters are strings to 
avoid enum fighting from people wanting to use non-standard values.
This does mean that you should be somewhat careful to check 
return values.

\end{description}






