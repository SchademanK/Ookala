
\section{Plugin Registration}

Essentially, we are building a plugin registration and
query system - so it makes sense to touch on the plugin 
registration process for a moment. Currently, registration 
is handled by the {\tt PluginRegistry} class. This class maintains
a list of plugin dsos, as well as object creation and deletion function
pointers. 

Currently, the plugin registry creates one instance of each 
plugin object. This makes book-keeping much easier, as other plugins
can query by name for a particular object. This does not, however, 
mean that plugins are singletons. Its entirely possible to maintain
multiple plugin registrys, each holding one distinct instance of
each registered plugin object. Forcing the plugins to be singletons 
at this point is overly oppressive for potential future expansion.

Plugin loading is handled in {\tt PluginRegistry::registerPlugins()}. 
This currently just takes a path to a module. A useful extension would
be a 'load everything from this path' method, with some logic for
re-loading to satisfy ordering constraints.

\subsection{Steps for making a loadable plugin}

As one might expect, to make a loadable plugin, we must create a 
class derived from {\tt Plugin}. The {\tt Plugin} class contains no pure
virtual methods, so if you are deriving straight from {\tt Plugin}, there
should not be an necessary methods to implement (however, 
other ``base'' classes may contain pure-virtual methods).

In order to deal with symbol searching and name mangling in dynamic
libraries, we need to jump through a few hoops. First, for each 
plugin class declaration, we need to use the macro
{\tt PLUGIN\_ALLOC\_FUNCS(className)}:
 
\begin{lstlisting}[frame=single]
    class foo: public Plugin
    {
        public:
           foo();
           virtual ~foo();

           PLUGIN_ALLOC_FUNCS(foo)
           ...
    };
\end{lstlisting}

Next, for each dynamic library, we need to provide a list of the
plugins contained within, that we wish to register.  This goes
somewhere around the definition of the plugins:

\begin{lstlisting}[frame=single]
 BEGIN_PLUGIN_REGISTER(numPlugins)
   PLUGIN_REGISTER(idx, pluginName)
 END_PLUGIN_REGISTER
\end{lstlisting}


This allows us to put mutiple plugins into a single dynamic library, 
for example:

\begin{lstlisting}[frame=single]
    class foo: public Plugin
    {
        public:
            foo() {
                mName.assign("foo");
            }

            virtual ~foo() {}

            PLUGIN_ALLOC_FUNCS(foo)

        protected:
            virtual bool _postLoad();
            virtual bool _checkDeps();
    };
    class bar: public Plugin
    {
        public:
            bar() {
                mName.assign("bar");

                mAttributes.push_back("tonic");
                mAttributes.push_back("olives");
            }

            virtual ~bar() {}

            PLUGIN_ALLOC_FUNCS(bar)
    };
    BEGIN_PLUGIN_REGISTER(2)
      PLUGIN_REGISTER(0, foo)
      PLUGIN_REGISTER(1, bar)
    END_PLUGIN_REGISTER 
\end{lstlisting} 

Which, when compiled into a dynamic library and loaded in {\tt PluginRegistry}
will provide two plugins, {\tt foo} and {\tt bar}.



\subsection{Plugin Queries}

The {\tt PluginRegistry} class also provides a mechanism for querying
plugins. Each plugin class has a name and a list of attributes. We
can query either by name, a single attribute, or a list of attributes,
and return a matching plugin. If multiple plugins are found to match,
we will get one that satisfies our needs:

\begin{lstlisting}[frame=single]
 std::vector<Plugin *> PluginRegistry::queryByName(const std::string name)

 std::vector<Plugin *> PluginRegistry::queryByAttribute(
               const std::string attrib)

 std::vector<Plugin *> PluginRegistry::queryByAttributes(
               const std::vector<std::string> attribs)
\end{lstlisting}

Plugin names are assumed to never collide, so only multiple-matching
queries by attributes are supported.

The name and attributes of a plugin should be set in the constructor,
by filling in strings 
containing the name and supported attributes of the plugin in
{\tt Plugin::setName()} and {\tt Plugin::addAttribute()} .


\subsection{Accessing the Plugin Registry from a Plugin}

Often, plugins will wish to make use of the the query methods in 
{\tt PluginRegistry}. To enable this, {\tt Plugin} holds a pointer
to the {\tt PluginRegistry} that holds the instance of the plugin. 

The {\tt PluginRegistry} pointer is assigned during the plugin 
loading and registration phase, after 
{\tt Plugin:: postLoad()}
executes.

\subsection{Extending Plugin Methods}

All the methods on {\tt Plugin} of interest follow this convention; The
outward-facing method looks normal, while the ``guts'' of the
method fall in the \_-prefixed version. This is done in cases where
we need to control access to the ``guts'' or wrap them with
more complicated logic.

Generally, if you're creating derived plugin classes, you only 
should have to worry about the \_-prefixed versions of these functions
for implementing the bulk of the work.

\subsubsection{Load-time methods}

We have three methods that are related to plugin loading
and inter-plugin dependency checking. They will be called
in the {\tt PluginRegistry} class, after loading the dynamic 
library containing the plugins. Returning {\tt false} to {\tt PluginRegistry}
is a signal that the plugin should be unloaded and not accessible.

\begin{description}
\item[] \
\begin{lstlisting}[frame=single]
 virtual bool Plugin::postLoad()	
 virtual bool Plugin::_postLoad()
\end{lstlisting}

After being loaded, a plugin should be given a chance to fail. This 
would be useful for cases where the plugin supports a particular 
device, but that device does not exist on the system. If this
fails (returns {\tt false}), the plugin should be removed from the registry 
so it cannot be queried and used elsewhere.

{\tt postLoad()} is called from within {\tt PluginRegistry}, and is generally 
not something you should ever have to call.

{\tt postLoad()} is called {\em before} {\tt Plugin::mRegistry} is assigned.
As such, you should never attempt to access any other plugins in {\tt postLoad()}
or {\tt \_postLoad()}
This is enforced because the plugin you want may not be loaded yet. If you
need to check for other plugins, you should do so in {\tt checkDeps()}.

\item[] \
\begin{lstlisting}[frame=single]
 virtual bool Plugin::checkDeps()
 virtual bool Plugin::_checkDeps()
\end{lstlisting}

After loading all the plugins, and making sure that they really
can work to some minimal degree, we should make sure that any 
plugins that we rely on exist and are functioning properly.
This will be called from within {\tt PluginRegistry}, after testing
{\tt postLoad()} on all plugins. 

If we return {\tt false} to a 
{\tt PluginRegistry}, the plugin will be removed.
For any plugins that we require to successfully run, we need
to query the plugin registry ( {\tt Plugin::mRegistry} ) and call {\tt checkDeps()}
on the returned plugin. Note that since this process can recurse,
we should take care not to include anything that would cause problems
when run recursivly.

\item [] \ 
\begin{lstlisting}[frame=single]
 virtual bool Plugin::postCheckDeps()
 virtual bool Plugin::_postCheckDeps()
\end{lstlisting}

The final method in the plugin loading routine. It will be run on
every plugin loaded in {\tt PluginRegistry}.


This can be handy for breaking cyclical dependencies that would be
introduced if all checking for other plugin support was performed
in {\tt checkDeps()}. However, it should not be though of as a replacement
for {\tt checkDeps()} since it is not ment to recursely execute on dependencies.

\end{description}

\subsection{Example}

Building upon our example declaration of two plugin classes in a dso,
{\tt foo} and {\tt bar}, we will build up a few simple load-time checking 
functions.

\begin{lstlisting}[frame=single]
 // _postLoad will check that the necessary doomsday dev
 // is available. If we can't destroy the whole world, there
 // is no use continuing.
 bool foo::_postLoad() 
 {
     int      fd;
     DictItem item;

     fd = open("/dev/doomsday", O_RDWR);
     if (fd < 0) {
         perror("open():");
         return false;
     }
     close(fd);
     return true;
 }

 // The foo plugin requires both the doomsday dev and the
 // the bar plugin for successful execution. We've checked
 // for /dev/doomsday in _postLoad(), but now we need to 
 // check for the bar plugin
 bool foo::_checkDeps() 
 { 
     std::vector<Plugin *> plugins;
     DictItem item;

     // See if we can locate the plugin in the registry.
     plugins = mRegistry->queryByName("bar");   
     if (plugins.empty()) {
         fprintf(stderr, 
             "foo::_checkDeps() ERROR: Can't locate plugin bar\n");
         return false;
     }

     // If it exists, we need to make sure that it also has its
     // needs filled.
     if (plugins[0]->checkDeps() == false) {
         fprintf(stderr, 
             "foo::_checkDeps() ERROR: Plugin bar failed checkDeps()\n");
         return false;
     }

     return true;
 }
\end{lstlisting}

