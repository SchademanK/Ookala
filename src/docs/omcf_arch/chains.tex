
\section{Plugin Chains}
\label{sec:Plugin Chains}

In general, 
we can divide common plugins into two groups - utilities and runnables.
There isn't an explicit divide between the types of plugins, its
more implicit based on which methods they override. In fact, a single
plugin could fall into both categories.

Utility plugins provide some service and can be used by a variety
of other plugins. Examples of utility plugins include things like
colorimeters or a DDC/CI connection. 

Runnable plugins are those that ``do'' something, and can be grouped
together into a chain with other plugins to run some task. The chain
is then executed serially. Runnable plugins will make use of utility
plugins, but generally not the other way around. An example of a 
runnable plugin would be a calibration algorithm.

For these runnable plugins, we have methods for dealing with the chain
of execution. Utility plugins will probably not need to override these
methods.

\subsection{Grouping Runnable Plugins into Chains}

The {\tt PluginChain} class provides a mechanism for grouping together
runnable plugins into something useful. Each plugin chain has facilities
for building a chain of plugins, accessing the plugins in the chain, 
and storing dictionary data for the plugins to use.

In general, you should not have to worry much about the details of
building a {\tt PluginChain}, though you will likely make great use
of its helper functionality. 

Running a {\tt PluginChain} consists of a few passes over the list of
plugins. First, the {\tt preRun()} method is called on each plugin.
If this succeeds, we proceed to the next step of calling {\tt run()}
on each plugin. Finally, {\tt postRun()} is called. {\tt postRun()}
will always be called, regardless of the success of the {\tt preRun()} 
and {\tt run()} steps.

\subsection{Runnable Plugin Methods}

Each of these methods has access to the {\tt PluginChain} from 
which it is being executed. This affords the plugins two useful
pieces: querying other plugins in the chain and canceling 
the execution of the chain. 

Querying other plugins is especially useful for finding a gui
plugin to use. Gui plugins should be added to chains, as
they are runnable (generally showing and hiding the display in 
{\tt preRun()} and {\tt postRun()} respectively). The alternative
would be to query the {\tt PluginRegistry} to find a gui plugin.
However, there could potentially be multiple gui-providing plugins
loaded at one time, and we would need some way of arbitration. 
Instead, we include gui plugins in the chain, only including one
gui plugin per chain. This way multiple chains can be constructed
that have differing guis.

Plugins in the chain have two methods available through the 
{\tt PluginChain} dealing with canceling the execution of the
chain. First, {\tt PluginChain::cancel()} will set the cancel
flag and stop execution of further plugins in the chain. As 
a plugin writer, you should call this if something fatal happens
or if the user sends a cancel signal. Also, since plugins may 
yield to the main gui loop to allow for handling events (e.g. 
cancel-button clicks), {\tt PluginChain::wasCancelled()} should
be tested occasionally to see if processing should be aborted.


\begin{description}
\item[] \
\begin{lstlisting}[frame=single]
virtual bool Plugin::preRun(PluginChain *chain)
virtual bool Plugin::_preRun(PluginChain *chain)
\end{lstlisting}

Prior to running a chain, we should give nodes the opportunity
to do any per-run things. Here we might put things like 
showing a gui, connecting to gui signals, and so on. 

{\tt preRun()} will be called when a plugin chain is run. Returning
false will skip {\tt preRun()} for the remaining plugins in the chain,
skip {\tt run()} for all plugins in the chain, but execute {\tt postRun()}
for all plugins.


\item[] \
\begin{lstlisting}[frame=single]
virtual bool Plugin::run(PluginChain *chain)
virtual bool Plugin::_run(PluginChain *chain)
\end{lstlisting}

This would be the place to put the guts of what you want to do.
It will probably query for other plugins, either from the registry
or from the chain.

Returning {\tt false} will skip {\tt run()} for the remaining plugins in the chain
but still call postRun() on all plugins.

\item[] \
\begin{lstlisting}[frame=single]
virtual bool Plugin::postRun(PluginChain *chain)
virtual bool Plugin::_postRun(PluginChain *chain)
\end{lstlisting}

After a chain is run, we should give each plugin a chance to cleanup
anyhthing that it might have setup. 

\end{description}

\subsection{Example Chain Functionality}

As an example of what the various run-time methods may implement, 
we outline a simple calibration process detailing what might 
occur in each method.

\begin{center}
\begin{tabular}{rl}
   preRun: & Gui opens         \\
           & Colorimeter setup \\
           & Colorimeter init  \\ \hline
   run:    & Reset LUTs \\
           & Calibration loop - display colors and measure \\
           & Save calibration data \\
           & Update LUTs \\ \hline
   postRun:& Colorimeter shutdown\\
           & Gui closes \\
           & If canceled, reset state
\end{tabular}
\end{center}

\subsection{Queries for Plugins in a Chain}

Similar to the plugin query functionality in {\tt PluginRegistry}, we can
also query plugins that live in a particular chain. This can be useful if 
you wish to specify which plugins are used by other plugins, instead of
relying on whatever we find in the {\tt PluginRegistry}.

\begin{description}
\item[] \
\begin{lstlisting}[frame=single]
 Plugin *     PluginChain::queryByName(const std::string name);
 Plugin *     PluginChain::queryByAttribute(const std::string attrib);
 Plugin *     PluginChain::queryByAttributes(
                               const std::vector<std::string> attribs);
       
 bool PluginChain::queryByAttribute(const std::string attrib, 
                         std::vector<Plugin *>&matches);
 bool PluginChain::queryByAttributes(
                         const std::vector<std::string> attribs, 
                               std::vector<Plugin *>&matches);
\end{lstlisting}
\end{description}



\subsection{Plugin Chain Gang}

While chains of plugins are useful, we also need a way to keep track of
all the chains, much the way plugins are tracked by {\tt PluginRegistry}.
For this, we have the {\tt PluginChainGang} class. 

{\tt PluginChainGang} is derived from {\tt Plugin}, so there should be
only one instance per {\tt PluginRegistry}, and it is queryable through
the {\tt PluginRegistry}.

On load, {\tt PluginChainGang} loads up chain configurations specified 
in a file. On disk, chains are stored something like:

\lstset{language=XML,
		columns=fixed,
        showspaces=false,
        showstringspaces=false,
        basicstyle=\ttfamily,
		backgroundcolor=\color{backGray}}
\begin{lstlisting}[frame=single]
 <chain>
     <name>The Name Of the Chain</name>

     <plugin>Plugin 0</plugin>
     <plugin>Plugin 1</plugin>
     <plugin>Plugin 2</plugin>

     <dict>
         <dictitem type="bool">
             <name>option 0</name>
             <value>true</value>
         </dictitem>
         ...
     </dict>
 </chain>
\end{lstlisting}
The list of plugins, in order of execution can be specified, as well as 
as the configuration dictionary for the chain. Chains for which all plugins
were loaded successfully are stored. Query methods are available to see
the names of available chains. Run methods are available for starting
execution of a particular chain, and a cancel method is available for 
externally canceling the currently running chain.

\subsubsection{Periodic Chain Execution}

It is occasionally hand to have chains execution continuously at some
specified interval. For example, implementing a status monitor that polls
the current calibration state could be done this way. 

To make a chain execution periodically, add a double value to the chain
dictionary named ``period''. The value should be the execution interval
in seconds. Note that if another chain is executing when the timer 
expires, the period chain may not run.

{\bf XXX: We still need to figure out where to specify the file that this
data should be pulled from. Currently, it gets pulled from the string
attribute ``Chain Config Files'' in the global dictionary.}

{\bf XXX: We also need to add external chain addition methods.}

{\bf XXX: It would be handy to include chains that are run when the
          program kicks off}





